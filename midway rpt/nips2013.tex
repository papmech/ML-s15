\documentclass{article} % For LaTeX2e
\usepackage{nips13submit_e,times}
\usepackage{hyperref}
\usepackage{url}
%\documentstyle[nips13submit_09,times,art10]{article} % For LaTeX 2.09


\title{Formatting Instructions for NIPS 2013}


\author{
Jonathan Yee\\
\texttt{jyee1@andrew.cmu.edu} \\
\And
Lingzhang Jiang \\
\texttt{lingzhaj@andrew.cmu.edu} \\
}

% The \author macro works with any number of authors. There are two commands
% used to separate the names and addresses of multiple authors: \And and \AND.
%
% Using \And between authors leaves it to \LaTeX{} to determine where to break
% the lines. Using \AND forces a linebreak at that point. So, if \LaTeX{}
% puts 3 of 4 authors names on the first line, and the last on the second
% line, try using \AND instead of \And before the third author name.

\newcommand{\fix}{\marginpar{FIX}}
\newcommand{\new}{\marginpar{NEW}}

\nipsfinalcopy % Uncomment for camera-ready version

\begin{document}


\maketitle

\begin{abstract}
I dunno what to put here lol
\end{abstract}

\section{Background}
We are trying to apply different machine learning techniques to an Optical Character Recognition(OCR) problem, with the objective of obtaining a high level of accuracy comparable on a given data-set comparable to state-of-the-art techniques.

OCR can be analyzed using the noisy channel model, with the source being the human mind, and the channel being the handwriting technique. We want to determine the most likely letter given the handwriting. 

For this project, we intend to design a Hidden Markov Model. The hidden states of the HMM will be the letters the human is thinking of, and the observed output is the actual handwriting represented as a pixel vector. This HMM can be graphically represented as a Bayes net. The goal is to find the sequence of hidden states (i.e. letters) that maximize the joint probability of those characters and the pixel vectors. Various OCR techniques (neural networks, naive bayes, logistic regression) can be applied to determine the probabilities that a pixel vector is a certain letter. We also intend to use the Viterbi algorithm to determine the most likely sequence of letters.

\section{Related Work}

\section{Experiments}

\section{Results}

\subsection{Style}

\subsection{Citations within the text}

Citations within the text should be numbered consecutively. The corresponding
number is to appear enclosed in square brackets, such as [1] or [2]-[5]. The
corresponding references are to be listed in the same order at the end of the
paper, in the \textbf{References} section. (Note: the standard
\textsc{Bib\TeX} style \texttt{unsrt} produces this.) As to the format of the
references themselves, any style is acceptable as long as it is used
consistently.

As submission is double blind, refer to your own published work in the 
third person. That is, use ``In the previous work of Jones et al.\ [4]'',
not ``In our previous work [4]''. If you cite your other papers that
are not widely available (e.g.\ a journal paper under review), use
anonymous author names in the citation, e.g.\ an author of the
form ``A.\ Anonymous''. 


\subsection{Footnotes}

Indicate footnotes with a number\footnote{Sample of the first footnote} in the
text. Place the footnotes at the bottom of the page on which they appear.
Precede the footnote with a horizontal rule of 2~inches
(12~picas).\footnote{Sample of the second footnote}

\subsection{Figures}

All artwork must be neat, clean, and legible. Lines should be dark
enough for purposes of reproduction; art work should not be
hand-drawn. The figure number and caption always appear after the
figure. Place one line space before the figure caption, and one line
space after the figure. The figure caption is lower case (except for
first word and proper nouns); figures are numbered consecutively.

Make sure the figure caption does not get separated from the figure.
Leave sufficient space to avoid splitting the figure and figure caption.

You may use color figures. 
However, it is best for the
figure captions and the paper body to make sense if the paper is printed
either in black/white or in color.
\begin{figure}[h]
\begin{center}
%\framebox[4.0in]{$\;$}
\fbox{\rule[-.5cm]{0cm}{4cm} \rule[-.5cm]{4cm}{0cm}}
\end{center}
\caption{Sample figure caption.}
\end{figure}

\subsection{Tables}

All tables must be centered, neat, clean and legible. Do not use hand-drawn
tables. The table number and title always appear before the table. See
Table~\ref{sample-table}.

Place one line space before the table title, one line space after the table
title, and one line space after the table. The table title must be lower case
(except for first word and proper nouns); tables are numbered consecutively.

\begin{table}[t]
\caption{Sample table title}
\label{sample-table}
\begin{center}
\begin{tabular}{ll}
\multicolumn{1}{c}{\bf PART}  &\multicolumn{1}{c}{\bf DESCRIPTION}
\\ \hline \\
Dendrite         &Input terminal \\
Axon             &Output terminal \\
Soma             &Cell body (contains cell nucleus) \\
\end{tabular}
\end{center}
\end{table}


\subsubsection*{Acknowledgments}

Use unnumbered third level headings for the acknowledgments. All
acknowledgments go at the end of the paper. Do not include 
acknowledgments in the anonymized submission, only in the 
final paper. 

\subsubsection*{References}

References follow the acknowledgments. Use unnumbered third level heading for
the references. Any choice of citation style is acceptable as long as you are
consistent. It is permissible to reduce the font size to `small' (9-point) 
when listing the references. {\bf Remember that this year you can use
a ninth page as long as it contains \emph{only} cited references.}

\small{
[1] Alexander, J.A. \& Mozer, M.C. (1995) Template-based algorithms
for connectionist rule extraction. In G. Tesauro, D. S. Touretzky
and T.K. Leen (eds.), {\it Advances in Neural Information Processing
Systems 7}, pp. 609-616. Cambridge, MA: MIT Press.

[2] Bower, J.M. \& Beeman, D. (1995) {\it The Book of GENESIS: Exploring
Realistic Neural Models with the GEneral NEural SImulation System.}
New York: TELOS/Springer-Verlag.

[3] Hasselmo, M.E., Schnell, E. \& Barkai, E. (1995) Dynamics of learning
and recall at excitatory recurrent synapses and cholinergic modulation
in rat hippocampal region CA3. {\it Journal of Neuroscience}
{\bf 15}(7):5249-5262.
}

\end{document}
